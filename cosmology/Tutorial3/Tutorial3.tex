\documentclass[a4paper,12pt]{article}
\usepackage{latexsym}
\usepackage{graphicx}
\usepackage{epsfig}
\usepackage{float}
\usepackage{natbib}
\usepackage{listings}
\usepackage{amsmath}
\graphicspath{{./}}
\DeclareGraphicsExtensions{.eps}
\author{Howard Kinsman}
\title{Cosmology Tutorial 3}
\begin{document}
\maketitle

\section{}
Here goes:
\begin{flalign*}
&s^2=\left(ct\right)^2-x^2 \\
&=c^2\gamma^2\left(t'-\frac{v'x'}{c^2}\right)-\gamma^2\left(x'-v't'\right)^2 &\\
&=\gamma^2\left(ct'-\frac{v'x'}{c}\right)^2-\gamma^2\left(t'^2v'^2-2t'v'x'+x'^2\right) &\\
&=\gamma^2\left(c^2t'^2+\frac{v'^2x'^2}{c^2}-2t'v'x'\right)-\gamma^2\left(t'^2v'^2-2t'v'x'+x'^2\right) &\\
&=\gamma^2\left(c^2t'^2+\frac{v'^2x'^2}{c^2}\right)-\gamma^2\left(t'^2v'^2+x'^2\right) &\\
&=\gamma^2\left[\left(c^2t'^2-t'^2v'^2\right)-x'^2\left(1-\frac{v'^2}{c^2}\right)\right] &\\
&=\gamma^2\left[\left(1-\frac{v'^2}{c^2}\right)c^2t'^2-x^2\left(1-\frac{v'^2}{c^2}\right)\right] &\\
&=\gamma^2\left[\left(\frac{1}{\gamma^2}c^2t'^2-x'^2\frac{1}{\gamma^2}\right)\right] &\\
&=c^2t'^2-x'^2
\end{flalign*}
\section{}
In special relativity time doesn't have a preferred frame but in cosmology, with an homogeneous and isotropic universe, then all observers can agree on a preferred
frame e.g. the temperature of the cosmic microwave background which decreases with the age of the universe.
\section{}
\begin{flalign*}
&dA=\frac{R_0r}{1+z} &\\
&dL=R_0r(1+z)
\end{flalign*}
so combining these gives
\begin{flalign*}
&dA/dL=\frac{\frac{R_0r}{1+z}}{R_0r(1+z)} &\\
&=(1+z)^2
\end{flalign*}
Both angular diameter distance and luminosity distance both assume a static Euclidean universe. Angular diameter distance is the distance
an object would have to be for its measured angular size. Luminosity distance is the distance an object would have to be to measure its
known luminosity. Proper distance is the actual distance an object would be according to a moment in 
time and varies with expansion of the universe. Light travel distance is the time it took light to reach the observer multiplied by the speed of light.
For low values of z then these distances are similar but for higher z they are different. As z increases the proper distance increases but tends towards a maximum because
the observable universe is finite,
the luminosity distances increases indefinitely because the 1+z term can increase indefinitely, whilst the angular diameter distance reaches a maximum and then decreases,
because at great distances the universe was smaller and so the angular diameter occupied a larger fraction of the universe.
\section{}
Proper distance is the actual distance an object would be according to a moment in 
time and varies with expansion of the universe. It would be the actual distance measured if one were to actually pace it out!
\begin{flalign*}
&d_{pr}(t_0)=R_0(f) &\\
&d_pr(t)=Rf(r) &\\
&d_pr(t_0)=\frac{R_0}{R}d_{pr}(t) &\\
&V_r=\frac{d}{dt}d_{pr}=\dot{R}f(r) &\\
&V_r=\frac{\dot{R}}{R}d_pr
\end{flalign*}
At small scales the Hubble law breaks down due to perculiar velocities (movements not associated with expansion of universe) e.g. the Andromeda galaxy is moving towards the
Milky Way. At large scales the Hubble parameter is different as to what it is now, plus the curvature of the universe affects the results.
%\begin{thebibliography}{1}
%\end{thebibliography}
\end{document} 