\documentclass[a4paper,12pt]{article}
\usepackage{latexsym}
\usepackage{graphicx}
\usepackage{epsfig}
\usepackage{float}
\usepackage{natbib}
\usepackage{listings}
\graphicspath{{./}}
\DeclareGraphicsExtensions{.eps}
\author{Howard Kinsman}
\title{Cosmology Tutorial 2}
\begin{document}
\maketitle

\section{}
Gravitational mass is the mass in Newton's inverse square law i.e. $F=\frac{GMM}{r^2}$ (either the active or passive mass
under the influence of a gravitational field) whilst inertial mass is the mass in
Newton's 2nd law of motion ie. $F=Ma$ (the mass which resists motion). In Newtonian mechanics there is no obvious reason
or explanation as to why these are the same thing.
Two other problems with Newtonian mechanics are the difference in the predicted and observed precession of the 
perihelion of Mercury according to classical mechanics; and, as Newtonian mechanics rely on inertial frames, there is no
simple definition as to what an inertial frame is within the context of Newton's equations.

\section{}
Starting with energy $E=mc^2$ an object falling in a gravitational field will lose potential energy $E_p=mgh$ but gain
the opposite in kinetic energy so will now have energy $E=mc^2+mgh$. Converting this into photon energy and letting photon
rise back in gravitational field it would obviously have gained energy - not possible according to conservation of energy.
So redshift is given by
\begin{equation}
z=\frac{\delta\lambda}{\lambda}
\end{equation}
and can also be written as 
\begin{equation}
1+z=\frac{\lambda'}{\lambda}
\end{equation}
reversing the fraction now (as the photon is rising) the above is equivalent to
\begin{equation}
1+z=\frac{\nu'}{\nu}
\end{equation}
and as $E=h\nu$ we get
\begin{equation}
1+z=\frac{E'}{E}
\end{equation}
or
\begin{equation}
1+z=\frac{mc^2+mgh}{mc^2}
\end{equation} 
giving 
\begin{equation}
1+z=1+\frac{gh}{c^2}
\end{equation}
and finally 
\begin{equation}
z=\frac{gh}{c^2}
\end{equation}
It is the same result as the Equivalence Principle.

\section{}
Curved space is a universe where space (and time) deviates from a Euclidean metric. There are two main types of curvature for
a homogeneous and isotropic universe: positive curvature or spherical geometry e.g. a sphere where $C<2\pi a$ and $k>0$;
and negative curvature or hyperbolic geometry e.g. saddle shaped where $C>2\pi a$ and $k<0$.

\section{}
\begin{equation}
k=\frac{3}{\pi}\lim_{a\to 0}\left(\frac{2\pi a - C}{a^3}\right)
\end{equation}
and
\begin{equation}
C=2\pi x = 2\pi R sin\frac{a}{R}
\end{equation}
however
\begin{equation}
sin x \approx x-\frac{x^3}{3!}
\end{equation}
so that
\begin{equation}
C\approx 2\pi a\left(1-\frac{a^2}{6R^2}\right)
\end{equation}
giving
\begin{equation}
k=\frac{3}{\pi}\lim_{a\to 0}\left(\frac{2\pi a - 2\pi a\left(1-\frac{a^2}{6R^2}\right)}{a^3}\right)
\end{equation}
which simplifies to
\begin{equation}
k=\frac{3}{\pi}\lim_{a\to 0}\left(\frac{\frac{2\pi a^3}{6R^2}}{a^3}\right)
\end{equation}
The $a^3$ cancel giving
\begin{equation}
k=\frac{3}{\pi}\frac{2\pi}{6R^2}
\end{equation}
and finally $6\pi$ cancels leaving
\begin{equation}
k=\frac{1}{R^2}
\end{equation}

\section{}
As the coordinates $\theta$ and $\phi$ represent angles and we know that the universe is isotropic and looks the same from all
directions and so the distance cannot depend on direction and therefore the metric can be simplified into 
$ds^2=f(r)dr^2+r^2d\theta^2+r^2sin^2\theta d\phi^2$ where $f(r,\theta,\phi)$ only
depends on r or distance $f(r)$ and also that $f'(r)=f''(r)$.
As the universe is homogeneous then the curvature k will be the same everywhere in space and time so we can use Gauss'
formula to calculate k which ultimately gives
\begin{equation}
\frac{1}{f^2dr}\frac{df}{dr}=2kr
\end{equation}
Integrating the above gives
\begin{equation}
\int\frac{1}{f^2}df=k\int 2r dr
\end{equation}
\begin{equation}
-\frac{1}{f}=kr^2-C
\end{equation}
which finally gives
\begin{equation}
f(r)=\frac{1}{C-kr^2}
\end{equation}
and setting $C=1$ because $f=1$ and $k=0$ for flat space we arrive at
\begin{equation}
f(r)=\frac{1}{1-kr^2}
\end{equation}
k determines the curvature of the universe and hence whether the universe is flat ($k=0$), open ($k<0$) or closed ($k>0$).
%\begin{thebibliography}{1}
%\end{thebibliography}
\end{document} 