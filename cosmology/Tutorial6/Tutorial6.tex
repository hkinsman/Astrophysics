\documentclass[a4paper,12pt]{article}
\usepackage{latexsym}
\usepackage{graphicx}
\usepackage{epsfig}
\usepackage{float}
\usepackage{natbib}
\usepackage{listings}
\usepackage{amsmath}
\graphicspath{{./}}
\DeclareGraphicsExtensions{.eps}
\author{Howard Kinsman}
\title{Cosmology Tutorial 6}
\begin{document}
\maketitle
\section{}
\subsection{}
\begin{flalign*}
& dE=-PdV &\\
& d(\rho c^2V)=-PdV &\\
& d(\rho V)=\frac{-P}{c^2}dV &\\
& Vd\rho + \rho dV=\frac{-P}{c^2}dV &\\
& Vd\rho = \frac{-P}{c^2}dV-\rho dV &\\
& Vd\rho=\left(\frac{-P}{c^2}-\rho\right)dV &\\
& d\rho=-\left(\rho+\frac{P}{c^2}\right)\frac{dV}{V} 
\end{flalign*}
\subsection{}
$V\propto R^3$ so
\begin{flalign*}
& \frac{d\rho}{\left(\rho + \frac{P}{c^2}\right)}=\frac{-dV}{V} &\\
& =\frac{-dR^3}{R^3} &\\
& =-3\frac{dR}{R}
\end{flalign*}
$d\rho=\frac{d\rho}{dt}dt$ so
\begin{flalign*}
& \frac{\frac{d\rho}{dt}dt}{\left(\rho+\frac{P}{c^2}\right)}=-3\frac{dR}{R} &\\
& \frac{d\rho}{dt}dt=\left(\rho+\frac{P}{c^2}\right)\frac{-3dR}{R} &\\
& R\frac{d\rho}{dt}=-3\left(\rho+\frac{P}{c^2}\right)\frac{dR}{dt} &\\
& \dot{\rho}=-3\left(\rho+\frac{P}{c^2}\right)\frac{\dot{R}}{R} &\\
& -3\left(\rho+\frac{P}{c^2}\right)\frac{\dot{R}}{R}=0 &\\
& \dot{\rho}+\frac{3\dot{R}}{R}\left(\rho+\frac{P}{c^2}\right)=0
\end{flalign*}
\subsection{}
$P=0$ so
\begin{flalign*}
& \dot{\rho}=\frac{3\dot{R}}{R}\rho &\\
& \frac{\dot{\rho}}{\rho}=\frac{-3\dot{R}}{R} &\\
& \frac{1}{\rho}\frac{d\rho}{dt}=-3\frac{1}{R}\frac{dR}{dt}
\end{flalign*}
integrating gives
\begin{flalign*}
& ln(\rho)=-3ln(R)+C &\\
& \rho\propto R^{-3}
\end{flalign*}
\subsection{}
$\rho=\frac{3P}{c^2}$ so
\begin{flalign*}
& \dot{\rho}+\frac{\dot{R}}{R}\left(3\rho + \rho\right)=0 &\\
& \dot{\rho}+\frac{\dot{R}}{R}4\rho=0 &\\
& \frac{\dot{\rho}}{\rho}=\frac{-4\dot{R}}{R}=0 &\\
& ln(\rho)=-4ln(R)+C &\\
& \rho\propto R^{-4}
\end{flalign*}
\subsection{}
$\rho$ is constant by definition so
\begin{flalign*}
& \rho+\frac{P}{c^2}=0 &\\
& \frac{P}{c^2}=-\rho &\\
& P=-\rho c^2 &\\
\end{flalign*}
So we have negative pressure - as the universe expands work is done on cosmological constant fluid allowing the energy density to remain constant whilst volume of 
universe increases (from Liddle).
\section{}
Flat galaxy rotation curves - spiral galaxies rotate faster at their edges than predicted from their observed mass, indicating there is some 'hidden' mass.
\newline
Gravitational lensing - GR predicts that light is bent by gravity - the light from galaxies behind galaxy clusters is bent more than predicted by GR again indicating some 'hidden' mass.
\newline
X-Ray observations - very hot gas within galaxy cluseters have been observed in X-Ray and the mass calculated from luminosity and temperature. The observed mass of the galaxies and this hot gas
again does not account for the total mass of the cluster indicating that the hidden mass in non-baryonic.
\section{}
\begin{flalign*}
& n_b=\frac{\Omega_{b,0} \rho_{crit,0}}{m_H} &\\
& n_b=\frac{.04\times 10^{-23} g/m^{-3}}{1.67\times 10^{-24}g} &\\
& n_b=.24 m^{-3} &\\
\end{flalign*}
I had to go to Weinberg for this one! For photons we integrate Planck function to get Stefan-Boltzman law $u\propto T^4$
\begin{flalign*}
& n_{\gamma}=\frac{30\zeta(3)}{\pi^4}\frac{\alpha_{rad}T^4_{CMB}}{k_B T_{CMB}} &\\
& n_{\gamma}=.3702\frac{\alpha_{rad}T^3}{k_B} &\\
& n_{\gamma}=\frac{7.56\times 10^{-16} 2.7^3}{1.38\times 10^{-23}} &\\
& n_{\gamma}=4\times 10^8 m^{-3}
\end{flalign*}
The results are quite suprising considering the ratio of photons to baryons is about $10^9$ and we have already determined that we are now in a matter dominated universe. How can that be?
\begin{flalign*}
& \Omega_{dm}\approx .3-.04=.26 &\\
\end{flalign*}
so the mass of dark matter particles:
\begin{flalign*}
& \frac{.26\times 10^{-23}}{.24}=1.08\times 10^{-23} g
\end{flalign*}
\section{}
\begin{flalign*}
& M=\frac{V^2 R}{G} &\\
& M=\frac{150,000^2 \times 1.5\times 10^{20}}{6.67\times 10^{-11}} &\\
& M=5\times 10^40 kg = 2.5\times 10^{10} M_{\odot} &\\
& \frac{M}{L}=\frac{2.5 \times 10^{10}}{6.5\times 10^{10}}=.39 &\\
\end{flalign*}
so mass at 50kpc:
\begin{flalign*}
& .39\times 1.3\times 10^{11}=5\times 10^{10} M_{\odot}
\end{flalign*}
\section{}
\subsection{}
I'm not sure what assumptions are being made.
\subsection{}
The mass was measured using weak gravitational lensing from more distant galaxies which lie beyond the merging clusters. This favours dark matter because the during a merger the stellar component of the 
clusters is collisionless whilst the plasma component is not and so they become separated. By mapping the gravitational lensing the study shows that the gravitational potential does not follow
plasma. This implies therefore that there is some 'hidden' mass.
\subsection{}
The title is quite bold. I think the study is a good case for the existence of dark matter but is not necessarily 'direct empirical proof'.
%\begin{thebibliography}{1}
%\end{thebibliography}
\end{document} 