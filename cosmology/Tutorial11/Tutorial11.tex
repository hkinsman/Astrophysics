\documentclass[a4paper,12pt]{article}
\usepackage{latexsym}
\usepackage{graphicx}
\usepackage{epsfig}
\usepackage{float}
\usepackage{natbib}
\usepackage{listings}
\usepackage{amsmath}
\usepackage{gensymb}
\graphicspath{{./}}
\DeclareGraphicsExtensions{.jpg}
\author{Howard Kinsman}
\title{Cosmology Tutorial 11}
\begin{document}
\maketitle
\section{}
 \begin{flalign*}
& M=V\rho &\\
& V=\frac{4}{3}\pi r^3 &\\
& r=\left(\frac{M}{\frac{4}{3}\pi\rho}\right)^{\frac{1}{3}} &\\
& r=\left(\frac{2\times10^{45}}{\frac{4}{3}\pi 10^{-26}}\right)^{\frac{1}{3}} &\\
& r=6.2\times10^{22}m &\\
& = 11.7 Mpc
\end{flalign*}
\section{}
\begin{flalign*}
& R_{200,m}=\left(\frac{2\times10^{45}}{\frac{4}{3}\pi 2\times10^{-24}}\right)^{\frac{1}{3}} &\\
& =6.2\times10^{22} m &\\
& =2Mpc
\end{flalign*}
\begin{flalign*}
& R_{200,m}=\left(\frac{2\times10^{44}}{\frac{4}{3}\pi 2\times10^{-24}}\right)^{\frac{1}{3}} &\\
& =2.88\times10^{22} m &\\
& =0.9Mpc
\end{flalign*}
\section{}
The four main ways to detect galaxy clusters are:
\newline
X-ray emission from hot gas - space X-Ray telescopes e.g. Chandra. Disadvantage of this method is that it is distance dependent, so better for closer galaxies.
\newline
Sunyaev-Zel'dovich effect - galaxy clusters cause inverse Compton scattering of CMB photons. Advantage of this method is that it is independent of redshift and so
best for more distant galaxies.
\newline
Optical - galaxy clusters can be observed directly in the optical. Disadvantage of this is that there may be galaxies within the line of sight which are not part of the cluster and
have to be removed. Also distance dependent.
\newline
Gravitational lensing - galaxy clusters bend light from galaxies behind the cluster - both strong and weak lensing effects can be used. Also distance dependent i.e. harder to observe the more distant
a galaxy cluster is.
%\begin{thebibliography}{1}
%\end{thebibliography}
\end{document} 