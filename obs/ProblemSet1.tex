\documentclass[a4paper,12pt]{article}
\usepackage{latexsym}
\usepackage{graphicx}
\usepackage{epsfig}
\usepackage{float}
\usepackage{natbib}
\usepackage{listings}
\graphicspath{{./}}
\DeclareGraphicsExtensions{.eps}
\author{Howard Kinsman}
\title{Observational Astrophysics Problem Set 1}
\begin{document}
\maketitle
\section{Question 1}
In order to derive Wiens's Displacement Law we need to find the maximum of Planck's Law, to do this we set the derivative to be zero: 
\begin{equation}
\frac{dI_\lambda(T)}{d\lambda} = \frac{d}{d\lambda}\left(\frac{2hc^2}{\lambda^5}\left(e^{\frac{hc}{\lambda kT}}-1\right)^{-1}\right) = 0
\end{equation}
Using the product rule $duv=vdu+udv$ yields the following:
\begin{equation}
\frac{-10hc^2}{\lambda^6}\left(e^{\frac{hc}{\lambda kT}}-1\right)^{-1}-\frac{2hc^2}{\lambda^5}\left(\frac{-hc}{\lambda^2kT}\right)\left(e^{\frac{hc}{\lambda kT}}\right)\left(e^{\frac{hc}{\lambda kT}}-1\right)^{-2} = 0
\end{equation}
As we are setting the equation to zero we can divide out the multiplicative terms giving:
\begin{equation}
\frac{-2hc^2}{\lambda^6}\left(e^{\frac{hc}{\lambda kT}}-1\right)^{-1}\left[5-\frac{hc}{\lambda kT}e^{\frac{hc}{\lambda kT}}\left(e^{\frac{hc}{\lambda kT}}-1\right)^{-1}\right] = 0
\end{equation}
If we now substitute $x=\frac{hc}{\lambda kT}$ we obtain:
\begin{equation}
5-xe^x\left(e^x-1\right)^{-1}
\end{equation}
which gives:
\begin{equation}
xe^x = 5(e^x-1)
\end{equation}
This equation has to be solved numerically however when solved for wavelength in nm and temperature in Kelvin we get:
\begin{equation}
\lambda_{max}=\frac{hc}{x}\frac{1}{kT}=\frac{2.89776829x10^6 nm . K}{T}
\end{equation}

\section{Question 2}
Specific intensity is distance independent because it contains the solid angle subtended by the source. If the source is unresolved i.e. at great distances like most stars
then this solid angle cannot be measured and instead of measuring specific intensity we would be measuring flux.

\section{Question 3}
The probability that a measurement is smaller than average by 2.5$\sigma$ is given by:
\begin{equation}
\int_{-\infty}^{\mu+2,5\sigma}\frac{1}{\sigma\sqrt{2\pi}}exp\left(-\frac{\left(x-\mu\right)^2}{2\sigma^2}\right)dt
\end{equation}
From the error table we get .99959 so the probability can be calculated as $1-(.5+.99959/2)\approx .0002$ or $.02\%$

\section{Question 4}
$B-V=8.35-7.81=0.54$ so from the table the spectral type is likely to be F8 with an effective temperature of 6250K.
It's absolute magnitude $M_v=4$ and so it's distance can be calculated using $D=10^{\left[\left(m-M+5\right)/5\right]} pc$ which 
equals $10^{\left[\left(7.81-4+5\right)/5\right]}=57.8 pc$.
The luminosity can be calculated from the bolometric magnitude so $\left(3\times10^{28}\right)\times10^{-0.4M_{bol}}=3.6\times10^{26} W$.

\section{Question 5}
\subsection{Part a}
Flux is given by $F=\frac{L}{4\pi r^2} W/m^2$ so solar luminosity is $3.846\times10^{26}W$ making Vega luminosity $1.538\times10^{28}$. 
Therefore Vega flux at Earth is $\frac{1.538\times10^{28}}{\left(4\pi 2.47\times10^{17}\right)^2}=2.52\times10^{-7} W/m^2$.
\subsection{Part b}
Flux density is also given by the inverse square rule so again $F_lambda=\frac{L}{4\pi r^2}$ but units are different so converting to cgs units we get:
$1W=10^7 erg s^{-1}$. Luminosity of Vega $=3.846\times10^{33} erg s^{-1}$. $8pc=2.469\times10^{19} cm$.

\subsection{Part c}


\end{document}































