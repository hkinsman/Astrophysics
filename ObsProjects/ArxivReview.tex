\documentclass[a4paper,12pt]{article}
\usepackage{latexsym}
\usepackage{graphicx}
\usepackage{epsfig}
\usepackage{float}
\usepackage{natbib}
\usepackage{listings}
\graphicspath{{./}}
\DeclareGraphicsExtensions{.eps}
\author{Howard Kinsman}
\title{A review of 'Searching for Primordial Black Holes in the radio and X-ray sky'}
\begin{document}
\maketitle
'Searching for Primordial Black Holes in the radio and X-ray sky' is a short paper \citep{gaggero} published on Arxiv on 1st December 2016 - https://arxiv.org/pdf/1612.00457v1.pdf.
It describes a technique for modelling primordial black holes (PBHs) and then comparing the 
model with observational measurements in both the radio and X-ray of potential PBH candidates in the Milky Way.
The recent LIGO detection of gravitational waves \citep{abbott}  has inferred the existence of $\approx30M_{\odot}$ black holes and prompted 
the suggestion that these may be PBHs \citep{bird}. 
A PBH is a hypothetical black hole that supposedly formed from the very dense matter found during the universe's early expansion. 
The idea that PBHs can account for the dark matter in the universe is not new e.g. \citep{bertone1} and \citep{bertone2}.

The study focuses on the galactic centre of the Milky Way and, given current estimates of the mass of the inner bulge, in order 
to account for all the dark matter there should be approximately $10^9$ PBHs in that vicinity. As the galactic centre contains
much high energy gas it is probable that a large portion of these PBHs would form accretion disks which would 
emit radiation at a wide range of wavelengths. This paper is restricted to radio and X-ray.
This study modelled PBHs using a Monte Carlo simulation which is a computer simulation based on sampling many random 
events and is widely used in astrophysics. A model of a galaxy was then built populated by PBHs. The accretion rate, radio
and X-ray emissions were then calculated.
These results were then compared with observations. For the radio candidates a VLA (Very Large Array) survey was used
to identify possible PBH candidates. A total of 24 possible PBHs were identified. These were compared with X-Ray sources from Chandra.
However the paper suggests that these candidates have to be ruled out as their X-Ray and radio fluxes do not lie together on the fundamental plane - 
a empirical relation familiar to MSc students from the Galaxy module. 
X-Ray candidates were selected from a survey carried out by the NuSTAR space telescope - a total of 70 were found to
be potential PBHs. However when compared with the VLA survey again no matches were found and so all these candidates were ruled out.
The paper concludes by saying that their accretion model has added more constraints on the possibility for PBHs to account for dark matter and suggests
that PBHs may only account for less than 20\% of dark matter. The study suggests that this technique
could be utilised further with the new radio telescopes such as MeerKat (South Africa) and the planned Square Kilometer
Array (SKA - South Africa and Australia) which could detect fainter sources.

As a newcomer to this field I found the paper readable and the introduction explained the background and purpose of the 
paper well. Although the conclusion was ultimately a negative one, the authors went on to investigate how the 
study could be repeated with new planned telescopes. The references indicate quite a substantial literature review
for such a small paper. I found the paper to be concise and informative. The reader could easily consult the references
if any further information on this subject was needed. I couldn't find any references to whether it
has been accepted by any refereed journals. I just have one criticism and that for me there wasm't a clear explanation as how they
reached the conclusion that $\approx30M_{\odot}$ PBHs account for less than $20\%$ of dark matter in the Milky Way.

\begin{thebibliography}{1}
\bibitem[Gaggero et al.(2016)]{gaggero}D. Gaggero, G. Bertone, F. Calore, R. M. T. Connors, M. Lovell, S. Marko, E. Storm Searching for Primordial Black Holes in the radio and X-ray sky (2016), arXiv:1612.00457 
\bibitem[Abbott et al.(2016)]{abbott}B. P. Abbott et al. (Virgo, LIGO Scientific), Phys. Rev. Lett. 116, 061102 (2016), arXiv:1602.03837 
\bibitem[Bird et al.(2016)]{bird}S. Bird, I. Cholis, J. B. Munoz, Y. Ali-Haimoud, M. Kamionkowski, E. D. Kovetz, A. Raccanelli, and A. G. Riess, Phys. Rev. Lett. 116, 201301 (2016), ArXiv:1603.00464
\bibitem[Bertone et al.(2005)]{bertone1}G. Bertone, D. Hooper, and J. Silk, Phys. Rept. 405, 279 (2005), arXiv:hep-ph/0404175
\bibitem[Bertone(2010)]{bertone2}G. Bertone, Nature 468, 389 (2010), arXiv:1011.3532
\end{thebibliography}
\end{document} 